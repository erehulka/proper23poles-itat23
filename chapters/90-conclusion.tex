\section*{Conclusion}
\addcontentsline{toc}{chapter}{Conclusion}
\markboth{Conclusion}{Conclusion}

Proper (2,3)-poles are multipoles resulting from snarks by removing one vertex and severing an edge, not incident with the removed vertex. In our work, we analysed how the colourings of proper (2,3)-poles can look. Based on their colouring sets, we have divided them into six classes. For each of them, we have found an example, thus proving that each class can be obtainable.

One of the main goals was to explore and describe perfect proper (2,3)-poles. That is why we have also provided some constructions, specifically 6-poles, which, when performing a junction with some colourable proper (2,3)-pole from some colouring class, result in a new proper (2,3)-pole that is from another colouring class. This way, it is possible to incrementally use these constructions to get perfect proper (2,3)-poles from any colouring class.

We explored all snarks with girth at least five and at most 28 vertices. This amounts to a total of 3247 snarks. We also examined all proper (2,3)-poles resulting from them, which is precisely 3476400 of them. Based on our exploration, we formulated some propositions regarding the colouring of proper (2,3)-poles. These propositions provide the necessary and sufficient conditions for their specific colouring properties. Most of the propositions were about the question of colourability of the proper (2,3)-poles, but also about being part of some specific colouring class. In most cases, these propositions were connected to removable pairs of vertices or edges.

By looking at the data, we see that more than half of the explored proper (2,3)-poles are perfect. However, we have not fulfilled the goal of finding sufficient conditions for the proper (2,3)-pole to be perfect. Only one necessary condition was found and proved. This may be an interesting goal for further research; it may be possible to find sufficient conditions using the provided data, observations and propositions.

Also, we have explored all bicritical snarks with girth at least five and at most 32 vertices. This amounts to a total of 278 snarks, with 306396 proper (2,3)-poles resulting from them.

Another interesting question for further research may be finding snarks, which produce only colourable proper (2,3)-poles. In the explored snarks, we have found five of them. Based on our propositions, this problem may be connected to finding snarks which have most of their edge pairs or all of them unremovable.